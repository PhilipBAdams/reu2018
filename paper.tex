\documentclass[psamsfonts]{amsart}

% -------Packages---------
\usepackage{microtype}[final]
\usepackage[style=numeric]{biblatex}
\usepackage{hyperref}
\addbibresource{./bib.bib}
\usepackage{amssymb,amsfonts}
\usepackage[all,arc]{xy}
\usepackage{enumerate}
\usepackage{mathrsfs}

%--------Theorem Environments--------
%theoremstyle{plain} --- default
\newtheorem{thm}{Theorem}[section]
\newtheorem{cor}[thm]{Corollary}
\newtheorem{prop}[thm]{Proposition}
\newtheorem{lem}[thm]{Lemma}
\newtheorem{conj}[thm]{Conjecture}
\newtheorem{quest}[thm]{Question}
\newtheorem{prob}[thm]{Problem}

\theoremstyle{definition}
\newtheorem{defn}[thm]{Definition}
\newtheorem{defns}[thm]{Definitions}
\newtheorem{con}[thm]{Construction}
\newtheorem{exmp}[thm]{Example}
\newtheorem{exmps}[thm]{Examples}
\newtheorem{notn}[thm]{Notation}
\newtheorem{notns}[thm]{Notations}
\newtheorem{addm}[thm]{Addendum}
\newtheorem{exer}[thm]{Exercise}

\theoremstyle{remark}
\newtheorem{rem}[thm]{Remark}
\newtheorem{rems}[thm]{Remarks}
\newtheorem{warn}[thm]{Warning}
\newtheorem{sch}[thm]{Scholium}


\makeatletter
\let\c@equation\c@thm
\makeatother
\numberwithin{equation}{section}

\title{Undecidability and the structure of the Turing degrees}

\author{Philip Adams}

\date{\today}

\begin{document}

\begin{abstract}

  This paper explores the structure and properties of the Turing degrees, or
  degrees of undecidability. We introduce the Turing machine, an abstract model
  of computation, in order to develop the concepts of undecidability and Turing
  reduction. We demonstrate the technique of proof by reduction through a series
  of examples of undecidable problems related to context-free grammars. We then
  employ reducibility to consider a partial ordering on the set of Turing degrees, $\mathcal{D}$. Finally, we prove a variety of theorems
  related to the structure of $\mathcal{D}$. 

\end{abstract}

\maketitle

\tableofcontents

\section{Introduction}
Historical overview of the field, motivation for study
\cite{ambos-spies06:_degrees_unsol}
\cite{soare16_turin_comput}
\cite{soare1999history}
\cite{lerman16:_degrees_unsol}


\section{Turing Machines}
% Turing 1936?
\cite{turing37_comput_number_with_applic_to_entsc}
\section{Undecidability}


\begin{prob}[The Halting Problem]
  \cite{turing37_comput_number_with_applic_to_entsc}
  \cite{sipser13:_introd_theor_comput}
\end{prob}
\begin{thm}[Rice's Theorem] \cite{kozen99_autom}
  % Proof by reduction to the halting problem?
\end{thm}
\subsection{Incompleteness}
\begin{thm}[G\"odel's Incompleteness Theorem] 
  % Kleene 1943? Also shown in hopcroft 1979
  \cite{kleene43_recur_predic_quant}
\end{thm}
\section{Reducibility}
\cite{sipser13:_introd_theor_comput}
\cite{post44:_recur}

\cite{kleene80_introd}

\section{Context Free Grammars}

\begin{defn}
 A Context Free Grammar is...
\end{defn}

\begin{defn}
  A Pushdown Automata is...
\end{defn}

\begin{thm}[Equivalence of CFGs and Pushdown Automata]
  \cite{hopcroft07:_introd_autom_theor_languag_comput}
\end{thm}
\begin{thm}[Closure Properties]
  \cite{sipser13:_introd_theor_comput}
  Union, Concat, Star, intersection with regular, substitution
\end{thm}
\begin{prob}
Empty, finite
\end{prob}
\subsection{Undecidable Problems}
\begin{prob}[The Post Correspondence Problem]
\cite{hopcroft07:_introd_autom_theor_languag_comput}
\end{prob}
\begin{prob}
 Undecidability of ambiguity 
\end{prob}
\cite{greibach66:_unsol_recog_linear_contex_free_languag}
\cite{Hopcroft1969}
\begin{prob}
  Universality
\end{prob}
\begin{prob}
  Equality, inclusion
\end{prob}
\begin{prob}
  Disjointness
\end{prob}
\subsubsection{Problems decidable for Deterministic CFGs}
\cite{ginsburg65:_deter}
\begin{defn}
  A Deterministic Context Free Grammar is...
  \cite{sipser13:_introd_theor_comput}
\end{defn}

\begin{thm}[Closure Properties]
  \cite{sipser13:_introd_theor_comput}
  Complement
\end{thm}

\begin{prob}[The Equivalence Problem for Deterministic CFGs]
\cite{senizergues_det_pd_decid}
\end{prob}

\begin{prob}
  Universality
\end{prob}


\section{Turing Degrees}
\begin{defn}
  A Turing degree...
\cite{post44:_recur}
\cite{kleene54_upper_semi_lattic_degrees_recur_unsol}
\end{defn}

\begin{defn}
  The jump operator...
  
\end{defn}

\begin{defn}
  Computably Enumerable
\end{defn}

\begin{defn}
  Completeness
\end{defn}

\subsection{Properties and Structure}
\begin{prob}[Post's Problem]
  \cite{post44:_recur}
  % Marchenkov 1976
  % Independently by Friedberg 57, Mucnik 56/58 using priority method
  % Priority free solution Kucer 1986''
  % ``Natural'' solns don't exist
  \cite{Friedberg236}
\end{prob}

% Myhill 56 c.e.lattice

\begin{lem}
  D does not form a lattice
  \cite{kleene54_upper_semi_lattic_degrees_recur_unsol}
\end{lem}
\begin{thm}
  D forms an upper semi-lattice
  % Kleene-Post 54 Degrees upper semi-lattice, jump an operator on degrees
  \cite{kleene54_upper_semi_lattic_degrees_recur_unsol}
\end{thm}

\begin{thm}
  Existence of minimal degrees
  % Spector 56,Shoenfield 66 (Better)
  \cite{spector56_degrees_recur_unsol}
  \cite{shoenfield66_theor_minim_degrees}
\end{thm}

\begin{thm}
  The c.e. degrees are dense
  \cite{sacks64:_recur_enumer_degrees_dense}
\end{thm}

Homogeneity problems
% Feiner 1970, Shore 1979

Theory of D Undecidable

\subsection{Turing Degrees of Problems related to CFGs}
\cite{REEDY197577}




\section*{Acknowledgments}
It is a pleasure to thank my mentor, Ronno Das, for supervising this project and
providing valuable feedback and advice. I would also like to thank Peter May for
organizing this REU, and Daniil Rudenko for running the Apprentice Program.

\printbibliography

\end{document}


%%% Local Variables:
%%% mode: latex
%%% TeX-master: t
%%% End:
