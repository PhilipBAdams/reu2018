\documentclass[psamsfonts]{amsart}

% -------Packages---------
\usepackage{microtype}[final]
\usepackage[style=numeric]{biblatex}
\usepackage{hyperref}
\addbibresource{/home/philip/documents/text/references/bibliography.bib}
\usepackage{amssymb,amsfonts}
\usepackage[all,arc]{xy}
\usepackage{enumerate}
\usepackage{mathrsfs}
\usepackage{todonotes}

%--------Theorem Environments--------
%theoremstyle{plain} --- default
\newtheorem{thm}{Theorem}[section]
\newtheorem{cor}[thm]{Corollary}
\newtheorem{prop}[thm]{Proposition}
\newtheorem{lem}[thm]{Lemma}
\newtheorem{conj}[thm]{Conjecture}
\newtheorem{quest}[thm]{Question}
\newtheorem{prob}[thm]{Problem}

\theoremstyle{definition}
\newtheorem{defn}[thm]{Definition}
\newtheorem{defns}[thm]{Definitions}
\newtheorem{con}[thm]{Construction}
\newtheorem{exmp}[thm]{Example}
\newtheorem{exmps}[thm]{Examples}
\newtheorem{notn}[thm]{Notation}
\newtheorem{notns}[thm]{Notations}
\newtheorem{addm}[thm]{Addendum}
\newtheorem{exer}[thm]{Exercise}

\theoremstyle{remark}
\newtheorem{rem}[thm]{Remark}
\newtheorem{rems}[thm]{Remarks}
\newtheorem{warn}[thm]{Warning}
\newtheorem{sch}[thm]{Scholium}


\makeatletter
\let\c@equation\c@thm
\makeatother
\numberwithin{equation}{section}

\title{Undecidability and the structure of the Turing degrees}

\author{Philip Adams}

\date{\today}

\begin{document}

\begin{abstract}

  This paper explores the structure and properties of the Turing degrees, or
  degrees of undecidability. We introduce the Turing machine, an abstract model
  of computation, in order to develop the concepts of undecidability and Turing
  reduction. We demonstrate the technique of proof by reduction through a series
  of examples of undecidable problems related to context-free grammars. We then
  employ reducibility to consider a partial ordering on the set of Turing degrees, $\mathcal{D}$. Finally, we prove a variety of theorems
  related to the structure of $\mathcal{D}$. 

\end{abstract}

\maketitle


\tableofcontents

\section{Introduction}
Historical overview of the field, motivation for study
\cite{ambos-spies06:_degrees_unsol}
\cite{soare16_turin_comput}
\cite{soare1999history}
\cite{lerman16:_degrees_unsol}
\subsection{Decision Problems}


\section{Turing Machines}
% Turing 1936?
\cite{turing37_comput_number_with_applic_to_entsc}
\section{Undecidability}


\begin{prob}[The Halting Problem]
  \label{prob:halting}
  \cite{turing37_comput_number_with_applic_to_entsc}
  \cite{sipser13:_introd_theor_comput}
\end{prob}
\begin{thm}[Rice's Theorem] \cite{kozen99_autom}
  % Proof by reduction to the halting problem?
\end{thm}
\subsection{Incompleteness}
\begin{thm}[G\"odel's Incompleteness Theorem] 
  % Kleene 1943? Also shown in hopcroft 1979
  \cite{kleene43_recur_predic_quant}
\end{thm}
\subsection{Reducibility}
While it is possible to prove that a problem is undecidable directly, as in
problem~\ref{prob:halting}, it is often more convenient to prove undecidability
through comparison to problems which are already known to be undecidable. This
comparison takes place through the technique of Turing reduction.
\begin{defn}
  Given two decision problems $A$ and $B$, we say that $A$ is Turing reducible
  to $B$ if, given a machine $D_B$ that decides $B$, it is possible to construct
  a machine $D_A$ that decides $A$.
\end{defn}
So, we have that if $A$ is reducible to $B$, then $A$ can be no harder than $B$,
because any solution to $B$ also leads to a solution of $B$. So, if we have some
problem $B$ that we would like to prove is undecidable, we can do so by showing
that some problem $A$ which is already known to be undecidable is reducible to
$B$. Since $A$ cannot be harder than $B$, it follows that $B$ must also be
undecidable. Alternatively, if we would like to show that some problem $A$ is
decidable, it is sufficient to show that it is reducible to some problem $B$
that is known to be decidable.
\cite{sipser13:_introd_theor_comput}
\cite{post44:_recur}

\cite{kleene80_introd}

\section{Context Free Grammars}

\begin{defn}
 A Context Free Grammar is...
\end{defn}

\begin{defn}
  A Pushdown Automata is...
\end{defn}

\begin{thm}[Equivalence of CFGs and Pushdown Automata]
  \cite{hopcroft07:_introd_autom_theor_languag_comput}
\end{thm}
\begin{thm}[Closure Properties]
  \cite{sipser13:_introd_theor_comput}
  Union, Concat, Star, intersection with regular, substitution
\end{thm}
\begin{prob}
Empty, finite, containment within regular
\end{prob}
\subsection{Undecidable Problems}
\begin{prob}[The Post Correspondence Problem]
  Consider some alphabet $\Sigma$ and two finite lists of words over $\Sigma$
  denoted $A = a_1,\dots,a_n$ and $B = b_1, \dots, b_n$. Then, does there exist
some sequence of indices $(i_k),$ for $1 \leq k \leq K$ and for
  $K\geq 1$, and with $1 \leq i_k \leq n$ for all $k$ such that
  \[
    a_{i_1}\dots a_{i_K} = b_{i_1}\dots b_{i_K}?
  \]
  \begin{proof}[Solution]
    This problem is shown to be undecidable through a reduction to the halting
    problem. The proof involves \todo{Describe proof in detail}
    \cite{sipser13:_introd_theor_comput}
  \end{proof}
\end{prob}

The Post correspondence problem is a useful tool, because it allows us to
demonstrate the undecidability of problems without doing the complex reasoning
about Turing machines that the halting problem requires. We will now show the
undecidability of a variety of problems about context-free grammars through
reduction to the Post correspondence problem.

\begin{prob}[Disjointness]
  Given two context-free grammars $P,Q$, is $L(P)\cap L(Q) = \varnothing$?
  \begin{proof}[Proof of Undecidability]
    Consider the Post correspondence problem for two lists of words $A,B$. We
    construct two context free grammars from these lists as follows:
    \begin{equation*}
      \begin{split}
        G_A \rightarrow  &\;a_11 \\
        &\!\!\!\!\vdots \\
        G_A \rightarrow  &\;a_nn \\
        G_A \rightarrow  &\;a_1G_A1 \\
        &\!\!\!\!\vdots \\
        G_A \rightarrow  &\;a_nG_An \\
      \end{split}
      \qquad
      \begin{split}
        G_B \rightarrow  &\;a_11 \\
        &\!\!\!\!\vdots \\
        G_B \rightarrow  &\;a_nn \\
        G_B \rightarrow  &\;a_1G_B1 \\
        &\!\!\!\!\vdots \\
        G_B \rightarrow  &\;a_nG_Bn \\
      \end{split}
    \end{equation*}
    Then, we can observe that for some string $s$ to exist in both $L(G_A)$ and $L(G_B)$,
    it must be a solution to the Post correspondence for $A,B$. So, if $L(G_A)\cap
    L(G_B)=\varnothing$, then there are no solutions to the Post correspondence
    for $A,B$. So, it follows that the Post correspondence problem is reducible
    to the problem of the disjointness of context-free grammars, so since the
    Post correspondence problem is undecidable, it follows that the disjointness
    problem is undecidable.
  \end{proof}
\end{prob}

\begin{prob}[Universality]
  \label{prob:cfg:universality}
  Take some context-free grammar $G$ over an alphabet $\Sigma$. Then, is
  \[
    L(G) = \Sigma^*?
  \]
  \begin{proof}[Proof of Undecidability]
    
  \end{proof}
\end{prob}

\begin{defn}
An ambiguous grammar is...
\end{defn}

\begin{prob}[Ambiguity]
  Take some context free grammar $G$. Is $G$ ambiguous?
  \begin{proof}[Proof of Undecidability]
    Consider the Post correspondence problem for two lists of words
    $A,B$. Construct their corresponding grammars $G_A,G_B$. Now, consider the
    grammar
    \[
      G \rightarrow G_A \vert G_B.
    \]
    It follows that the ambiguity of $G$ implies that a solution to the Post
    correspondence problem for $A,B$ exists, so the Post correspondence problem
    reduces to the ambiguity problem, so the ambiguity problem is undecidable.
  \end{proof}
\end{prob}
\cite{greibach66:_unsol_recog_linear_contex_free_languag}
\cite{Hopcroft1969}
\begin{prob}
  Regularity
\end{prob}
\begin{prob}[Equality]
  Take context free grammars $G_1,G_2$. Is $L(G_1)=L(G_2)$?
  \begin{proof}[Proof of Undecidability]
    Note that $\Sigma^*$ is regular, so it is also a context-free. So,
    we have that Problem~\ref{prob:cfg:universality} reduces to the equality
    problem. It follows that the equality problem is undecidable.
  \end{proof}
\end{prob}
\begin{prob}[Inclusion]
  Take context free grammars $G_1,G_2$. Is $L(G_1)\subseteq L(G_2)$?
  \begin{proof}[Proof of Undecidability]
    As before, note that $\Sigma^*$ is regular, so it is also a
    context-free. Additionally, observe that for any grammar $G$, $L(G)\subseteq
    \Sigma^*$. So,
    we have that Problem~\ref{prob:cfg:universality} reduces to the inclusion
    problem, since $\Sigma^* \subseteq L(G) \implies \Sigma^* = L(G)$. It follows that the inclusion problem is undecidable.
  \end{proof}
\end{prob}

\subsubsection{Problems decidable for Deterministic CFGs}
\cite{ginsburg65:_deter}
\begin{defn}
  A Deterministic Context Free Grammar is...
  \cite{sipser13:_introd_theor_comput}
\end{defn}

\begin{thm}[Closure Properties]
  \cite{sipser13:_introd_theor_comput}
  Complement
\end{thm}

\begin{prob}[The Equivalence Problem for Deterministic CFGs]
\cite{senizergues_det_pd_decid}
\end{prob}

\begin{prob}
  Universality
\end{prob}


\section{Turing Degrees}
\begin{defn}
  A Turing degree...
\cite{post44:_recur}
\cite{kleene54_upper_semi_lattic_degrees_recur_unsol}
\end{defn}

\begin{defn}
  The jump operator...
  
\end{defn}

\begin{defn}
  Computably Enumerable
\end{defn}

\begin{defn}
  Completeness
\end{defn}

\subsection{Properties and Structure}
\begin{prob}[Post's Problem]
  \cite{post44:_recur}
  % Marchenkov 1976
  % Independently by Friedberg 57, Mucnik 56/58 using priority method
  % Priority free solution Kucer 1986''
  % ``Natural'' solns don't exist
  \cite{Friedberg236}
\end{prob}

% Myhill 56 c.e.lattice

\begin{lem}
  D does not form a lattice
  \cite{kleene54_upper_semi_lattic_degrees_recur_unsol}
\end{lem}
\begin{thm}
  D forms an upper semi-lattice
  % Kleene-Post 54 Degrees upper semi-lattice, jump an operator on degrees
  \cite{kleene54_upper_semi_lattic_degrees_recur_unsol}
\end{thm}

\begin{thm}
  Existence of minimal degrees
  % Spector 56,Shoenfield 66 (Better)
  \cite{spector56_degrees_recur_unsol}
  \cite{shoenfield66_theor_minim_degrees}
\end{thm}

\begin{thm}
  The c.e. degrees are dense
  \cite{sacks64:_recur_enumer_degrees_dense}
\end{thm}

Homogeneity problems
% Feiner 1970, Shore 1979

Theory of D Undecidable

\subsection{Turing Degrees of Problems related to CFGs}
\cite{REEDY197577}




\section*{Acknowledgments}
It is a pleasure to thank my mentor, Ronno Das, for supervising this project and
providing valuable feedback and advice. I would also like to thank Peter May for
organizing this REU, and Daniil Rudenko for running the Apprentice Program.

\printbibliography

\end{document}


%%% Local Variables:
%%% mode: latex
%%% TeX-master: t
%%% End:
